\documentclass[12pt,onecolumn,a4paper]{article}
\usepackage{epsfig,amsthm,amsmath,booktabs,csquotes}
\usepackage [pagebackref=true, colorlinks, linkcolor=blue, citecolor=magenta, urlcolor=cyan] {hyperref}
\usepackage{color,xcolor}

\usepackage{subcaption}
\usepackage[labelformat=parens,labelsep=quad, skip=3pt]{caption}
\usepackage{graphicx}
\usepackage{enumerate}
\usepackage[localise]{xepersian}
\settextfont[Scale=1.2]{BZAR.TTF}
\setlatintextfont[Scale=1]{Times New Roman}





\begin{document}
\title{کارگاه برنامه نویسی آشنایی با گیت‌هاب} 
\author{محسن مهرانی}
\date{\today}
\maketitle
\قسمت{سیر من در آموختن گیت}

\begin{enumerate}[1.]
\item
تریسته\پانویس{Trieste, Italy}- بهار ۹۸: اولین جایی که با گیت آشنا شدم.
\item
تهران\پانویس{Tehran, Iran}- تابستان ۹۸: اولین پروژه برنامه نویسی مشترک رو با چهار تن از دوستانم کلید زدیم و به پایان رساندیم.
\item
تهران - بهار ۱۴۰۰: میزبان کارگاه آموزشی آن شدم.
\item
تهران؟ - ۱۴۰۰؟: گسترش پروژه‌های گروهی دوستان دعوت شده در کارگاه و ...
\item ...
\end{enumerate}

\قسمت{فواید و کاربردها}
گیت این توانایی را به ما می‌دهد که کنترل بیشتری روی پروژه‌های «متنی» خودمان از جمله کدهای برنامه نویسی یا مقالات داشته باشیم. گیت به ما این امکان رو می‌ده که:
\begin{enumerate}
\item
نسخه‌های متفاوت از کدهامان را به صورت منظم نگهداری کنیم و نیازی به فولدرسازی‌های پی در پی با اسامی مختلف نداشته باشیم.
\item
هر موقع ایده‌ای به ذهنمان رسید بدون نگرانی روی یک نسخه پشتیبان ایده‌مان را امتحان کنیم و در صورت تایید به کد اصلی اضافه کنیم.
\item
کارگروهی را به خوبی و بدون تداخل مدیریت کنیم.
\item
و بسیاری ویژگی‌های دیگر ...
\end{enumerate}

\قسمت{اقدامات لازم پیش از شروع کارگاه}
برای شروع کارگاه لازم است از انجام شدن دو کار مطمئن شوید:
\زیرقسمت{نصب نرم افزار $Gitbash$}
لطفا آخرین نسخه سازگار با دستگاه رایانه خود را از 
\href{https://git-scm.com/downloads}{این پایگاه}
بارگیری و نصب بفرمایید.
\زیرقسمت{ایجاد حساب کاربری در تارنمای $Github$}
به تارنمای 
\href{https://github.com/}{گیت‌هاب}
 مراجعه و با رایانامه خود ثبت نام کنید. پس از ثبت نام لطفا نام کاربری خودتان را به آدرس رایانامه من به آدرس زیر ارسال کنید. به عنوان مثال نام کاربری من در گیت‌هاب $mmehrani$ است و می‌توانید با مراجعه به آدرس $github.com/mmehrani$ صفحه من را مشاهده کنید.\\
 \begin{flushleft}
 $m.mehrani14@gmail.com$
 \end{flushleft}
\قسمت{پیشنهادات}
\زیرقسمت{مطالعه در حالت تاریک}
اگر شما هم مثل من ساعت‌ها پشت رایانه خود می‌نشینید و مطالعه می‌کنید؛ پیشنهاد می‌کنم حتما افزونه‌ی $Dark\, reader$ را که برای مرورگرها طراحی شده، امتحان کنید.

\href{https://chrome.google.com/webstore/detail/dark-reader/eimadpbcbfnmbkopoojfekhnkhdbieeh}{پیوند بارگیری برای مرورگر کروم}

\زیرقسمت{هم‌گروهی سازگار}
ما در اینجا کارگروهی را باهم تمرین خواهیم کرد. روانشناسان برای شناساندن انسان‌ها به خودشان تست‌هایی را طراحی می‌کنند که یکی از معروف‌ترین آنها تست $MBTI$ است. روانشناسان صنعتی و مدیران موفق معمولا از این تست برای انتخاب و تشکیل کارگروه‌هایی متشکل از افراد سازگار با یکدیگر استفاده می‌کنند.\\
 مثلا من تیپ شخصیتی
 $
 INFJ \footnote{$Introverted-iNtiutive-Feeling-Judging$}
 $
  دارم و با کسانی که تیپ شخصیتی دارند که شامل دو حرف $NF$است؛ همگروهی خوبی می‌شوم
  \footnote{البته قول نمی‌دهم:)}
  . پیشنهاد می‌کنم قبل از تشکیل یک گروه منسجم حتما پیشنهادات و مطالب موجود در 
  \href{https://www.16personalities.com/fa/}{این تارنما}
  را مطالعه کنید.
\end{document}


